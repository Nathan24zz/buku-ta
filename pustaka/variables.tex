% Atur variabel berikut sesuai namanya

% nama
\newcommand{\name}{Nathanael Hutama Harsono}
\newcommand{\authorname}{Harsono, Nathanael Hutama}
\newcommand{\nickname}{Nathan}
\newcommand{\advisor}{Prof. Dr. Ir. Mauridhi Hery Purnomo, M.Eng.}
\newcommand{\coadvisor}{Dion Hayu Fandiantoro, S.T.,M.T.}
\newcommand{\examinerone}{Dr. Galileo Galilei, S.T., M.Sc}
\newcommand{\examinertwo}{Friedrich Nietzsche, S.T., M.Sc}
\newcommand{\examinerthree}{Alan Turing, ST., MT}
\newcommand{\headofdepartment}{Dr. Supeno Mardi Susiki Nugroho, S.T., M.T.}

% identitas
\newcommand{\nrp}{0721 19 4000 0044}
\newcommand{\advisornip}{19580916 198601 1 001}
\newcommand{\coadvisornip}{1994202011064}
\newcommand{\examineronenip}{18560710 194301 1 001}
\newcommand{\examinertwonip}{18560710 194301 1 001}
\newcommand{\examinerthreenip}{18560710 194301 1 001}
\newcommand{\headofdepartmentnip}{18810313 196901 1 001}

% judul mimikri pose tubuh bagian atas manusia oleh robot humanoid menggunakan fungsi cosine similarity
\newcommand{\tatitle}{\emph{MIMIKRI POSE TUBUH BAGIAN ATAS MANUSIA OLEH} ROBOT HUMANOID \emph{MENGGUNAKAN KESAMAAN KOSINUS}}
\newcommand{\engtatitle}{UPPER BODY HUMAN POSE MIMICRY BY HUMANOID ROBOT USING COSINE SIMILARITY FUNCTION}

% tempat
\newcommand{\place}{Surabaya}

% jurusan
\newcommand{\studyprogram}{Teknik Komputer}
\newcommand{\engstudyprogram}{Computer Engineering}

% fakultas
\newcommand{\faculty}{Teknologi Elektro dan Informatika Cerdas}
\newcommand{\engfaculty}{Intelligent Electrical and Informatics Technology}

% singkatan fakultas
\newcommand{\facultyshort}{FTEIC}
\newcommand{\engfacultyshort}{ELECTICS}

% departemen
\newcommand{\department}{Teknik Komputer}
\newcommand{\engdepartment}{Computer Engineering}

% kode mata kuliah
\newcommand{\coursecode}{EC224801}
