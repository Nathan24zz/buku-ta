\chapter{CONCLUSION}
\label{chap:conclusion}

In this chapter, the conclusions from the test results will be presented which will be the answer to the problems raised by this research.
In addition, recommendations for actions that could be taken can advance this research in a further direction will also be presented.

\section{Summary}
\label{sec:summary}

Based on the results of the study, we can make the following conclusion:

\begin{enumerate}[nolistsep]

  \item The creation of a new dataset has been completed by merging the HumanoidRobotPose dataset and Ichiro's poses dataset 
        so the size of the robots become varies (adult, teen, and kid size). Furthermore, most of the additions (Ichiro's pose dataset) are one robot instance and
        few are two instances per image with large scale (according to COCO dataset).
  \item RCNN Keypoint is capable and reliable for detecting humanoid robot with 6 keypoints because it outperforms other models in \emph{AP} and \emph{AR} results on the test set (except for AP medium)
        and shows the best detection results although it can not be performed in real-time even after converting to OpenVINO.
  \item Cosine Similarity is a suitable method to compare the suitability between humanoid robot pose with human poses because the result is higher when
        human make a movement that is more like a robot's motion and the result is lower when human make a movement that is less like robot's motion.

\end{enumerate}

\section{Suggestions And Future Work}
\label{chap:suggestionsandfuturework}

For future development of "Mimicking Between Humanoid Robot and Human Based on Real-Time Pose Estimation" we have a few suggestions:

\begin{enumerate}[nolistsep]

  \item Add teen-sized humanoid robot to the dataset so that the types of robots without skin can be varied.
  \item Try to add more keypoints on humanoid robot so the result of pose comparison more precise.

\end{enumerate}
