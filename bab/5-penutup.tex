\chapter{PENUTUP}
\label{chap:conclusion}

Dalam bab ini, kesimpulan dari hasil pengujian akan disajikan sebagai jawaban terhadap masalah yang diangkat dalam penelitian ini.
Selain itu juga terdapat rekomendasi untuk tindakan yang dapat dilakukan dalam mengembangkan penelitian ini ke arah yang lebih lanjut.

\section{Kesimpulan}
\label{sec:summary}

Berdasarkan hasil pengujian yang telah dilakukan, penulis dapat menyimpulkan beberapa hal sebagai berikut:

\begin{enumerate}[nolistsep]

  \item Pembuatan dataset baru telah diselesaikan dengan mengkombinasikan HumanoidRobotPose dataset dan dataset milik kita sendiri
        sehingga ukuran robot dalam dataset menjadi bervariasi (\textit{adult, teen, kid size}).
        Sebagian besar penambahan adalah data dengan konfigurasi 1 robot tiap gambar dan ukuran \textit{large} (sesuai dengan COCO Dataset)
  \item RCNN Keypoint adalah metode yang mampu untuk mendeteksi pose robot \textit{humanoid} dengan 6 \textit{keypoint} karena mengungguli model lain,
        dengan hasil AP sebesar 0.879 dan AR sebesar 0.925 pada test set (kecuali untuk AP \textit{medium}).
  \item RCNN Keypoint is capable and reliable for detecting humanoid robot with 6 keypoints because it outperforms other models in \emph{AP} and \emph{AR} results on the test set (except for AP medium)
        and shows the best detection results although it can not be performed in real-time even after converting to OpenVINO.
  \item Kesamaan Kosinus merupakan metode yang cocok untuk membandingkan pose robot dan manusia karena hasilnya akan lebih tinggi
        ketika manusia melakukan gerakan yang lebih menyerupai gerakan robot dan hasilnya akan lebih rendah ketika gerakan manusia tidak semirip dengan gerakan robot.

\end{enumerate}

\section{Saran}
\label{chap:suggestionsandfuturework}

Untuk pengembangan lebih lanjut pada penelitian mendatang, maka penulis memiliki saran sebagai berikut:

\begin{enumerate}[nolistsep]

  \item Menambahkan robot \textit{humanoid} dengan ukuran \textit{teen} ke dalam dataset sehingga tipe dari robot tanpa kulit dapat lebih beragam.
  \item Mencoba menambah \textit{keypoint} pada robot \textit{humanoid} sehingga hasil perbandingan pose lebih akurat.

\end{enumerate}
