\chapter{INTRODUCTION}
\label{chap:introduction}

% Ubah bagian-bagian berikut dengan isi dari pendahuluan
This study is motivated by various conditions that become a reference. Apart from that, there are also some problems that
will be answered as an outcome of the study.

\section{Background}
\label{sec:background}

Robots have experienced significant development over the last few years 
because of their ability to perform multiple tasks quickly and precisely.
One form of development is socially assistive robots (SARs). 
SARs are a type of robot that combines the aspects of assistive robotics (AR)
and socially interactive robotics (SIR), so it makes SARs a robot capable of providing assistance to users in the form of social interaction \parencite{feil2005}.

Regular physical activity is a central protective factor for health.
A report from WHO says that lack of physical activity contributes to around 3.2 million premature deaths each year worldwide.
The research also shows that regular exercise can help older adults improve physical fitness, immune system, sleep quality, stress levels, and overcome other health problems \parencite{lotfi2018}.
However, motivation to engage in physical activity declines with age. 
This is due to understaffed and high costs of personal trainers as well as the elderly who cannot be permanently motivated and instructed to engage in physical activity.
Based on research conducted by \parencite{ruf2020} it can be concluded that the use of humanoid robot can motivate older people to carry out regular physical activity.

The application of SARs is very diverse, for example, humanoid robot that become physical trainers for children \parencite{güneysu2017}, 
robotic systems for physical training of the elderly \parencite{avioz2021}, and so on. The methods used also vary, for instance, 
you can put sensors on the user and get feedback, then provide a response based on the data obtained through the sensor \parencite{güneysu2017}. 
Another method is to use pose estimation obtained through \emph{deep learning} models.
However, previous studies have only compared the angle and depth of each keypoint from estimated human pose within certain boundaries or compared the results of human poses with the poses that are used as a guide. \parencite{romeo}
There is still little study that directly compares the suitability between poses performed by human and robot.

For that, on this occasion, we propose study related to mimicking between humanoid robot and human based on real-time pose estimation. 
In addition to finding the best pose estimation method for both humanoid robot and human, this study will also compare poses between them and make a web for controlling their interaction.
This is intended so that in the future it can be implemented in real robot or in an educational game.

\section{Problem}
\label{sec:problem}

The problem that can be drawn are there have been studies on pose estimation for robot with skin but no one has done it on robot without skin (for example Ichiro's robot).
There are not many studies that directly compare the estimated poses of humanoid robot with human. 
For this reason, a study is needed to know the suitability of humanoid robot poses and human poses. 
In addition, it is also necessary to add Ichiro's pose dataset to the HumanoidRobotPose dataset and find the best pose estimation model for humanoid robot and human.

\section{Goal}
\label{sec:goal}

The purpose of this study is to create a new dataset that merges Ichiro's poses dataset and the HumanoidRobotPose dataset so it can increase the diversity of the data. In addition,
we will look for the best pose estimation method for the humanoid robot. Third, comparing the suitability of humanoid robot pose with human poses.

\section{Study Limits}
\label{sec:studylimits}

The limitations of the problem in focusing on the problems formulated in this study are:

\begin{enumerate}[nolistsep]

  \item Merging NimbRo's HumanoidRobotPose dataset and Ichiro's dataset to form a new dataset that is more diverse.
  \item Doing a test using the new dataset to look for the best robot pose estimation algorithm from the following methods: RCNN, NimbRo, and YOLO.
  \item Looking for human pose estimation detection algorithms that best match with robot poses include: OpenPose, MediaPipe, and YOLO.
  \item Training is carried out on the ITS supercomputer (DGX) and then implemented on computer devices with limited computing capabilities (e.g. NUC i5)

\end{enumerate}

\section{Writing System}
\label{sec:writingsystem}

This final project study is arranged in a systematic manner.
It is structured so that it is easy for readers to understand and learn.
This final project study is divided into several sections as follows:

\begin{enumerate}[nolistsep]

  \item \textbf{CHAPTER I Introduction}

        Chapter 1 of this book contains a description of the background, problem raised, goal, limitations of the study, and writing system.

        \vspace{2ex}

  \item \textbf{CHAPTER II Literature Study}

        Chapter 2 describes supporting theories related to this study, such as descriptions of pose estimation, Human and Humanoid Robot Pose Estimation, Robot Operating Systems (ROS), and so forth.

        \vspace{2ex}

  \item \textbf{CHAPTER III Design and System Implementation}

        Chapter 3 describes system design and implementation, starting from collecting data to forming new datasets, training pose estimation models for humanoid robot,
        searching for the best model for both human and humanoid robot, and comparing the suitability between them.

        \vspace{2ex}

  \item \textbf{CHAPTER IV Results and Discussion}

        Chapter 4 consists of the results obtained from the tests performed, such as comparing humanoid robot models performance based on \emph{mAP}, \emph{mAR}, time inference,
        and real detection result.

        \vspace{2ex}

  \item \textbf{CHAPTER V Conclusion}

        Conclusions from the study and suggestions for further research are described in chapter 5 in this book.

\end{enumerate}
