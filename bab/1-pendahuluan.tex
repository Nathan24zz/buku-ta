\chapter{PENDAHULUAN}
\label{chap:pendahuluan}

% Ubah bagian-bagian berikut dengan isi dari pendahuluan
Bab ini menjelaskan mengenai garis besar tugas akhir yang meliputi latar belakang, 
rumusan masalah, batasan masalah, tujuan, dan manfaat tugas akhir. Dari penjelasan yang 
ditulis dalam bab ini, diharapkan gambaran umum pada tugas akhir kali ini dapat dipahami.

\section{Latar Belakang}
\label{sec:latar-belakang}

Robot telah mengalami perkembangan yang signifikan selama beberapa tahun terakhir karena kemampuannya untuk melakukan banyak tugas dan pekerjaan secara cepat dan tepat. Salah satunya adalah \emph{socially assistive robots} (SARs). SARs merupakan jenis robot dalam bidang \emph{socially assistive robotics} yang menggabungkan aspek yang ada pada \emph{assistive robotics} dan \emph{socially interactive robotics} sehingga menjadikan SARs sebagai robot yang mampu memberikan bantuan kepada pengguna dalam bentuk interaksi sosial \parencite{feil2005}.

Aktivitas fisik secara teratur merupakan salah satu faktor yang penting bagi kesehatan. Sebuah laporan dari \emph{WHO} menyebutkan bahwa kurangnya aktivitas fisik itu berkontribusi pada sekitar 3,2 juta kematian dini setiap tahun di seluruh dunia.
Penelitian tersebut juga menunjukkan bahwa olahraga secara teratur dapat membantu orang tua dewasa untuk dapat meningkatkan kebugaran fisik, sistem kekebalan tubuh, kualitas tidur, tingkat stres dan mengatasi masalah-masalah kesehatan lainnya \parencite{lotfi2018}. Namun, motivasi untuk melakukan aktivitas fisik menurun seiring bertambahnya usia. Hal ini disebabkan oleh kurangnya staf dan tingginya biaya pelatih pribadi serta orang lanjut usia yang tidak dapat dimotivasi secara permanen serta diinstruksikan untuk terlibat dalam aktivitas fisik. Berdasarkan penelitian yang dilakukan oleh \parencite{ruf2020} dapat disimpulkan bahwa penggunaan robot humanoid dapat memotivasi orang lanjut usia untuk melakukan aktivitas fisik secara reguler.

Penerapan dari SARs sangatlah beragam, contohnya seperti humanoid robot yang menjadi pelatih fisik bagi anak-anak \parencite{güneysu2017},
sistem robot untuk pelatihan fisik orang lanjut usia \parencite{avioz2021}, dan lain sebagainya. Metode yang digunakan juga bermacam-macam,
contohnya dapat mengenakan sensor pada pengguna dan mendapatkan umpan balik, kemudian memberikan respon berdasarkan data yang didapat melalui sensor tersebut \parencite{güneysu2017}.
Metode lainnya adalah memnggunakan estimasi pose yang didapat melalui model \emph{deep learning}.
Akan tetapi penelitian terdahulu hanya membandingkan sudut dan kedalaman dari setiap sendi atau \emph{keypoint} dari estimasi pose manusia (pengguna) dengan batasan tertentu atau membandingkan hasil pose manusia dengan pose yang dijadikan panduan \parencite{romeo}.
Masih sedikit penelitian yang membandingkan secara langsung kecocokan antara pose yang dilakukan oleh manusia dan pose yang dicontohkan oleh robot secara langsung.

Untuk itu, pada kesempatan ini kami mengajukan penelitian terkait imitasi pose tubuh bagian atas manusia oleh robot \emph{humanoid} menggunakan kesamaan kosinus.
Selain mencari metode estimasi pose terbaik dan membandingkan kecocokan antara estimasi pose robot \emph{humanoid} dan pose manusia, penilitian ini juga membuat website untuk mengontrol interaksi diantara keduanya.

\section{Permasalahan}
\label{sec:permasalahan}

Dari latar belakang yang telah dipaparkan sebelumnya, maka permasalahan yang dapat
diambil adalah belum banyaknya studi yang membandingkan secara langsung estimasi pose
humanoid robot dengan manusia. Untuk itu, diperlukan studi yang membandingkan kecocokan
estimasi pose robot humanoid dengan pose manusia. Selain itu, diperlukan juga adanya penambahan dataset pose robot Ichiro pada HumanoidRobotPose dataset serta mencari model estimasi
pose terbaik untuk humanoid robot dan manusia.

\section{Tujuan}
\label{sec:tujuan}

Tujuan dari penelitian ini adalah membuat dataset pose robot ichiro untuk ditambahkan
pada HumanoidRobotPose dataset sehingga dapat menambah keberagaman data. Selain itu,
menggunakan dataset yang sudah ditambahkan sebelumnya, kami akan mencari metode estimasi pose terbaik bagi humanoid robot. Ketiga, membandingkan kecocokan estimasi pose
robot humanoid dengan pose manusia.

\section{Batasan Masalah}
\label{sec:batasan-masalah}

Batasan-batasan masalah serta ruang lingkup penelitian yang dilakukan berdasarkan
permasalahan yang ada adalah sebagai berikut

\begin{enumerate}[nolistsep]

  \item Menggabungkan HumanoidRobotPose dataset milih NimbRo dan dataset Ichiro untuk membentuk dataset baru yang lebih beragam.
  \item Melakukan pengujian terhadap dataset baru untuk mencari algoritma pose estimasi terbaik robot dari beberapa metode berikut: Model NimbRo, Keypoint RCNN, dan YOLO-pose.
  \item Mencari algoritma deteksi estimasi pose manusia yang paling cocok dengan pose robot meliputi: OpenPose, MediaPipe, YOLO-pose.
  \item Proses training dilakukan pada superkomputer ITS (DGX) dan diimplementasikan pada perangkat komputer dengan kemampuan komputasi yang terbatas (misal NUC i5).   

\end{enumerate}

\section{Manfaat}
\label{sec:manfaat}

Manfaat dari penilitian ini adalah:

\begin{enumerate}[nolistsep]

  \item \textbf{Bagi penulis}

      Seluruh proses yang dijalani penulis dalam pelaksanaan perancangan Tugas Akhir ini
      memberikan banyak manfaat kepada penulis, antara lain seperti mengasah kepekaan
      terhadap permasalahan nyata yang dihadapi oleh masyarakat, pola pikir yang kritis, logis
      dan sistematis dalam perumusan sintesis pemecahan masalah, ilmu-ilmu pengetahuan
      baru yang menambah wawasan penulis, keterampilan dalam menyusun tugas akhir ini
      hingga akhir dan lain sebagainya.

        \vspace{2ex}

  \item \textbf{Bagi Institusi}
      Melalui laporan Tugas Akhir ini, penulis berharap dapat memberikan inspirasi dan
      referensi terkait inovasi - inovasi dalam bidang \emph{deep learning} khususnya dalam bidang pose estimasi.
      Sehingga perancangan ini tidak berhenti sampai disini saja namun
      dapat terus berkembang dan menjadi suatu sistem yang sempurna di masa yang akan datang.
        
        \vspace{2ex}

  \item \textbf{Bagi Masyarakat}

      Penulis berharap bahwa melalui pembuatan Tugas Akhir ini dapat menjadi sebuah solusi yang
      tepat sasaran baik dari aspek permasalahan maupun kebutuhan masyarakat sehingga dapat
      memberikan manfaat yang nyata dalam kehidupan sehari-hari.
  
\end{enumerate}
