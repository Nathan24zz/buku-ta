\chapter{PENDAHULUAN}
\label{chap:pendahuluan}

% Ubah bagian-bagian berikut dengan isi dari pendahuluan
This research is motivated by various conditions that become a reference. Apart from that, there are also some problems that
will be answered as an outcome of the research.

\section{Latar Belakang}
\label{sec:latarbelakang}

Robots have experienced significant development over the last few years 
because of their ability to perform multiple tasks quickly and precisely.
One form of development is socially assistive robots (SARs). 
SARs are a type of robot that combines the aspects of assistive robotics (AR)
and socially interactive robotics (SIR), so it makes SARs a robot capable of providing assistance to users in the form of social interaction \parencite{feil2005}.

Regular physical activity is a central protective factor for health.
A report from WHO says that lack of physical activity contributes to around 3.2 million premature deaths each year worldwide.
The research also shows that regular exercise can help older adults improve physical fitness, immune system, sleep quality, stress levels, and overcome other health problems \parencite{lotfi2018}.
However, motivation to engage in physical activity declines with age. 
This is due to understaffed and high costs of personal trainers as well as the elderly who cannot be permanently motivated and instructed to engage in physical activity.
Based on research conducted by \parencite{ruf2020} it can be concluded that the use of humanoid robots can motivate older people to carry out regular physical activity.

The application of SARs is very diverse, for example, humanoid robots that become physical trainers for children \parencite{güneysu2017}, 
robotic systems for physical training of the elderly \parencite{avioz2021}, and so on. The methods used also vary, for instance, 
you can put sensors on the user and get feedback, then provide a response based on the data obtained through the sensor \parencite{güneysu2017}. 
Another method is to use pose estimation obtained through \emph{deep learning} models.
However, previous studies have only compared the angle and depth of each keypoint from estimated human pose within certain boundaries or compared the results of human poses with the poses that are used as a guide. \parencite{romeo}
There is still little research that directly compares the suitability between poses performed by humans and robots.

For that, on this occasion, we propose research related to mimicking between humanoid robots and humans based on real-time pose estimation. 
In addition to finding the best pose estimation method for both humanoid robots and humans, this research will also compare poses between them and make a web for controlling their interaction.
This is intended so that in the future it can be implemented in real robots or in an educational game.

\section{Permasalahan}
\label{sec:permasalahan}

The problem that can be drawn are there have been studies on pose estimation for robots with skin but no one has done it on robots without skin (for example Ichiro's robot).
There are not many studies that directly compare the estimated poses of humanoid robots with humans. 
For this reason, a study is needed to know the suitability of humanoid robot poses and human poses. 
In addition, it is also necessary to add Ichiro's pose dataset to the HumanoidRobotPose dataset and find the best pose estimation model for humanoid robots and humans.

\section{Tujuan}
\label{sec:Tujuan}

The purpose of this research is to create a new dataset that merges Ichiro's poses dataset and the HumanoidRobotPose dataset so it can increase the diversity of the data. In addition,
we will look for the best pose estimation method for the humanoid robot. Third, comparing the suitability of humanoid robot pose with human poses.

\section{Batasan Masalah}
\label{sec:batasanmasalah}

The limitations of the problem in focusing on the problems formulated in this research are:

\begin{enumerate}[nolistsep]

  \item Merging NimbRo's HumanoidRobotPose dataset and Ichiro's dataset to form a new dataset that is more diverse.
  \item Doing a test using the new dataset to look for the best robot pose estimation algorithm from the following methods: RCNN, NimbRo, and YOLO.
  \item Looking for human pose estimation detection algorithms that best match with robot poses include: OpenPose, MediaPipe, and YOLO.
  \item Training is carried out on the ITS supercomputer (DGX) and then implemented on computer devices with limited computing capabilities (e.g. NUC i5)

\end{enumerate}

\section{Sistematika Penulisan}
\label{sec:sistematikapenulisan}

This final project research is arranged in a systematic manner.
It is structured so that it is easy for readers to understand and learn.
This final project research is divided into several sections as follows:

\begin{enumerate}[nolistsep]

  \item \textbf{BAB I Pendahuluan}

        Bab ini berisi \lipsum[2][1-5]

        \vspace{2ex}

  \item \textbf{BAB II Tinjauan Pustaka}

        Bab ini berisi \lipsum[3][1-5]

        \vspace{2ex}

  \item \textbf{BAB III Desain dan Implementasi Sistem}

        Bab ini berisi \lipsum[4][1-5]

        \vspace{2ex}

  \item \textbf{BAB IV Pengujian dan Analisa}

        Bab ini berisi \lipsum[5][1-5]

        \vspace{2ex}

  \item \textbf{BAB V Penutup}

        Bab ini berisi \lipsum[6][1-5]

\end{enumerate}
