\begin{center}
  \Large
  \textbf{KATA PENGANTAR}
\end{center}

\addcontentsline{toc}{chapter}{KATA PENGANTAR}

\vspace{2ex}

% Ubah paragraf-paragraf berikut dengan isi dari kata pengantar

Puji dan syukur ke hadirat Tuhan YME atas segala limpahan berkah, penulis dapat
menyelesaikan penelitian ini dengan judul: \textbf{"Imitasi Pose Tubuh Bagian Atas Manusia oleh Robot Humanoid Menggunakan Kesamaan Kosinus"}.

Penelitian ini disusun dalam rangka pemenuhan bidang riset di Departemen Teknik Komputer, serta digunakan sebagai
persyaratan menyelesaikan pendidikan S1. Dalam penyusunan buku ini, penulis ingin mengucapkan terima kasih kepada beberapa
pihak yang telah memberikan dukungan dan bantuan dalam penyelesaian tugas akhir ini.
Penulis ingin menyampaikan ucapan terimakasih sebesar-besarnya kepada:

\begin{enumerate}[nolistsep]

  \item Dr. Supeno Mardi Susiki Nugroho, ST., MT. selaku Kepala Departemen Teknik Komputer, Fakultas Teknik Elektro dan Informatika Cerdas, Institut Teknologi Sepuluh Nopember.
  \item Prof.Dr.Ir. Mauridhi Hery Purnomo, M.Eng. selaku dosen pembimbing I yang telah memberikan arahan dan bimbingan selama pengerjaan penelitian ini.
  \item Dion Hayu Fandiantoro, S.T.,M.T. selaku dosen pembimbing II yang telah memberikan arahan dan bimbingan selama pengerjaan penelitian ini.
  \item Muhtadin, S.T., M.T. selaku dosen pembimbing Tim ICHIRO ITS yang telah memberikan arahan dan bimbingan selama pengerjaan penelitian ini.
  \item Bapak-ibu dosen pengajar Departemen Teknik Komputer, atas pengajaran, bimbingan, serta perhatian yang diberikan kepada penulis selama ini
  \item Orang tua penulis yang telah memberikan dorongan spiritual dan material dalam penyelesaian tugas akhir ini.
  \item Rekan penulis, Amik Rafly Azmi Ulya, David Antonius Setiawan, dan Segara Bhagas Dagsapurwa yang selalu mendukung penulis baik dalam hal akademik maupun non-akademik.
  \item Seluruh teman-teman dari Tim ICHIRO ITS, angkatan E59, Laboratorium B401 dan B201 Teknik Komputer ITS.
\end{enumerate}

Serta pihak-pihak lain yang tidak dapat disebutkan satu-persatu. Semoga laporan tugas akhir ini dapat dipergunakan dengan sebaik-baiknya. Penulis menyadari bahwa masih terdapat banyak kekurangan pada laporan tugas akhir ini,
sehingga penulis mengharapkan adanya saran dan masukan yang membangun agar dapat menjadi lebih baik lagi kedepannya. Semoga laporan tugas akhir ini dapat dipergunakan dengan sebaik-baiknya.

\begin{flushright}
  \begin{tabular}[b]{c}
    \place{}, \MONTH{} \the\year{} \\
    \\
    \\
    \\
    \\
    \name{}
  \end{tabular}
\end{flushright}
