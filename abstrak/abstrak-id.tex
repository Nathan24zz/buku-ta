\begin{center}
  \large\textbf{ABSTRAK}
\end{center}

\addcontentsline{toc}{chapter}{ABSTRAK}

\vspace{2ex}

\begingroup
% Menghilangkan padding
\setlength{\tabcolsep}{0pt}

\noindent
\begin{tabularx}{\textwidth}{l >{\centering}m{2em} X}
  Nama Mahasiswa    & : & \name{}         \\

  Judul Tugas Akhir & : & \tatitle{}      \\

  Pembimbing        & : & 1. \advisor{}   \\
                    &   & 2. \coadvisor{} \\
\end{tabularx}
\endgroup

% Ubah paragraf berikut dengan abstrak dari tugas akhir
Kebiasaan untuk melakukan aktivitas fisik secara teratur merupakan faktor yang penting bagi kesehatan. 
Namun, terkadang motivasi untuk melakukan aktivitas fisik menurun seiring bertambahnya usia.
Hal ini dapat diatasi dengan menggantikan posisi pelatih fisik dengan \emph{robot humanoid}. 
Perbedaan penelitian ini dengan penelitian-penelitian terdahulu adalah membandingkan pose \emph{robot humanoid} dengan manusia secara langsung, 
yang nantinya akan digunakan pada proses \emph{mimicking}. 

Sebelum itu, akan dibentuk dataset baru yang diperoleh melalui panggabungan  dataset HumanoidRobotPose milik NimbRo dengan dataset pose robot Ichiro. 
Setelah itu, mencari metode estimasi terbaik bagi \emph{robot humanoid} dan manusia. 
Melalui penelitian ini, diharapkan adanya dataset pose \emph{robot humanoid} baru serta program pencocokan antara pose humanoid robot dengan manusia.

% Ubah kata-kata berikut dengan kata kunci dari tugas akhir
Kata Kunci: \emph{Aktivitas fisik, Mimicking, Pose}.
