\begin{center}
  \large\textbf{ABSTRACT}
\end{center}

\addcontentsline{toc}{chapter}{ABSTRACT}

\vspace{2ex}

\begingroup
% Menghilangkan padding
\setlength{\tabcolsep}{0pt}

\noindent
\begin{tabularx}{\textwidth}{l >{\centering}m{3em} X}
  \emph{Name}     & : & \name{}         \\

  \emph{Title}    & : & \engtatitle{}   \\

  \emph{Advisors} & : & 1. \advisor{}   \\
                  &   & 2. \coadvisor{} \\
\end{tabularx}
\endgroup

% Ubah paragraf berikut dengan abstrak dari tugas akhir dalam Bahasa Inggris
\emph{The habit of doing regular physical activity is a central protective factor for health.
However, sometimes the motivation to engage in physical activity decreases with age.
Luckily, this can be overcome by replacing the position of a physical trainer with a humanoid robot.
The main difference between this research and some previous research is comparing humanoid robot's pose with human's pose directly, 
which will be used in the mimicking process.}

\emph{Before that, there will be a new dataset obtained by merging of NimbRo's HumanoidRobotPose dataset with Ichiro's robot pose dataset.
After that, looking for the best pose estimation method for humanoid robots and humans.
Through this research, it is hoped that there will be a new dataset of humanoid robot pose as well as a program for matching between humanoid robot's pose and human's pose.}

% Ubah kata-kata berikut dengan kata kunci dari tugas akhir dalam Bahasa Inggris
\emph{Keywords}: \emph{Physical activity}, \emph{Mimicking}, \emph{Pose}.
